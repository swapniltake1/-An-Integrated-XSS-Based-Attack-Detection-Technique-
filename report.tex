\documentclass[12pt]{report}
\usepackage{graphicx}
\usepackage{xcolor}
\usepackage{setspace}
\usepackage {fancybox}
\usepackage{fancyhdr}
\usepackage{enumitem}
\pagestyle{fancy}
\fancyhf{}
\lfoot{\thepage}
\rfoot{
}
\usepackage{titlesec}
\titleformat{}
  {\centering \normalfont \LARGE \bfseries}
  \centering {Chapter \thechapter }{1em} {}
    %\newcommand{\cchapter}[1]{\chapter[#1]{\centering #1}}
% \chapterfont{\centering}
\usepackage[a4paper,vmargin={30mm,20mm},hmargin={40mm,30mm}]{geometry}
\begin{document}
\thisfancypage{\setlength{\fboxsep}{10pt}\doublebox}{}
\clearpage\thispagestyle{empty}
\begin{flushright}\end{flushright}



 \begin{center}
   \begin{textblock*}
   \noindent\hspace{0.0in}\small{\textbf{A SEMINAR REPORT ON}}

\vspace*{0.2in}

\noindent\vspace{0.0in}\large{\textbf{\color{blue}"Intelligent Internet Of Things Service Based on Artificial
Intelligence Technology"}}
   
\vspace*{0.2in}
\textbf{\small{SUBMITTED TO}}   
   
\vspace*{0.2in}
\textbf{\large{SAVITRIBAI PHULE PUNE UNIVERSITY}}  
   
\vspace*{0.2in}
\textbf{\small{As Per}}

\vspace*{0.2in}
\textbf{\large{COMPUTER ENGINEERING}} 

\vspace*{0.2in}
\textbf{\small{Submited By}}

\vspace*{0.2in}
\textbf{\large{ Munjal Bhushan Balasaheb  }}

\vspace*{0.2in}
\textbf{\small{Roll No : 25 }}

\vspace*{0.2in}
\textbf{\small{Under The Guidance Of}}

\vspace*{0.2in}
\textbf{\large{Prof.Ghadge R.A.}}

\vspace*{0.2in}
\includegraphics[width=4cm, height=4cm]{collagelogo.png}

\vspace*{0.2in}
\textbf{\large{DEPARTMENT OF COMPUTER ENGINEERING}}

\vspace*{0.05in}
\textbf{\small{VISHWABHARATI ACADEMY'S COLLAGE OF ENGINEERING}}

\vspace*{0.05in}
\textbf{\small{SAROLA BADDI}}

\vspace*{0.05in}
\textbf{\small{AHMEDNAGAR 414201}}

\vspace*{0.2in}
\textbf{\small{2021-2022}}
   
    \end{textblock*}

\end{center}
\newpage
\vspace{1 cm}
\clearpage\thispagestyle{empty}
\thisfancypage{\setlength{\fboxsep}{10pt}\doublebox}{}


\begin{center}
 

\includegraphics[width=4cm, height=4cm]{collagelogo.png} \\
\vspace{0.5in}
\fontsize{14pt}{16pt}\selectfont \textbf{DEPARTMENT OF COMPUTER ENGINEERING}\\
\fontsize{12pt}{14pt}\selectfont \textbf{Vishwabharati Academy's Collage Of Engineering \\
Sarola Baddi,Ahmednagar}\\[0.7in]
\fontsize{18pt}{18pt}\selectfont \textbf {CERTIFICATE }\\[0.5in]
\fontsize{12pt}{14pt}\selectfont   
\begin{flushleft}
This is to certify that the Mr. Munjal Bhushan Balasaheb from TE Computer Engineering, Roll no. 25  submitted his seminar report on \textbf{"Intelligent Internet Of Things Service Based on
Artificial Intelligence Technology"} under my guidance  Prof. Ghadge R.A. and supervision. The work has been done to my satisfaction during the academic year 2021-2022 under Savitribai Phule Pune University. 
\end{flushleft}
\vspace{1.5in}
\begin{flushleft}

%Prof. Pachhade.R.C \hspace{1.0cm} Prof.Pachhade.R.C \hspace{1.5cm} Prof.Joshi.S.G \\
 %\hspace{12.3cm}Bhandari\\
%\textbf{Seminar Guide \hspace{2cm}  Seminar Co-ordinator \hspace{2cm} H.O.D  

\centering{
\hspace{0.1cm} Prof.Ghadge R.A. \hspace{5.0cm} Prof.Sapike N.S.  
\centering \textbf{\hspace{3.5cm} Seminar Guide    \hspace{4.0cm} Seminar Co-ordinator}
\paragraph{}
\vspace{0.2in}            
\hspace{0.5cm} Prof.Joshi S.G. \hspace{5.0cm} Prof.Dhongade V.S.   \textbf{\hspace{5.5cm} H.O.D            \hspace{7.0cm} Principal} }

\end{flushleft}
\vspace{0.15in}
\begin{flushleft}
Date :\\
\vspace{0.1in}
Place :Ahmednagar\\
\end{flushleft}
\end{center}

\newpage

\chapter* { \centering Acknowledgement}\thisfancypage{\setlength{\fboxsep}{10pt}\doublebox}{}
\pagenumbering{Roman}
\vspace{1cm}
\par
\onehalfspacing
\paragraph{}
This is a great pleasure \& immense satisfaction to express my deepest sense of gratitude
\& thanks to everyone who has directly or indirectly helped me in completing my Seminar
work successfully.\\

I express my gratitude towards seminar guide  {\bfseries Prof. Ghadge R.A.} and {\bfseries Prof.Joshi S. G.} Head Of Department of Computer
Engineering, Vishwabharati Academy's Collage Of Engineering, Sarola Baddi, Ahmednagar, who guided \& encouraged me in completing the Seminar work in scheduled time.\\

I would like to thanks our Principal {\bfseries Prof.Dhongade V. S. }, for his extended support.
No words are sufficient to express my gratitude to my family for their unwavering
encouragement. I also thank all friends for being a constant source of my support.\\
\vspace{3cm}
\begin{flushright}{Mr.Munjal Bhushan Balasaheb}\end{flushright}
\begin{flushright}{T. E. Computer}\end{flushright}
\begin{flushright}{Roll No: 25}\end{flushright}

\tableofcontents
\listoffigures
%\addcontentsline{toc}{chapter}{List of Figures}
%\addcontentsline{toc}{chapter}{List of Tables}



\chapter*{\centering Abstract}
\paragraph{} The global Internet industry is becoming more and more mature, showing vigorous
development potential and vitality, and gradually becoming an important part of promoting
economic development and challenges. In the future, with the help of artificial intelligence
technology, big data technology, 5G and other new technologies, and with the increase of Internet
usage, the focus of Internet connection will shift from people oriented connection to Internet of
things connection. Since 2020, the novel coronavirus pneumonia outbreak has spread worldwide,
and it has brought great challenges to the global economy and society. The Internet has played an
important role in the resumption of work and production, economic recovery and maintenance of
social operation. 
\par The application of the Internet of things has alleviated some negative effects of
the epidemic, and has made an important contribution to economic development. In 2020, China’s
Internet technology presents a new high level of loading, network infrastructure construction is
accelerating, 5G network is built and used, and emerging infrastructure such as big data centers
are built and applied in various cities. At present, China’s network information technology
self-control ability is gradually enhanced, and breakthroughs have been made in big data
technology, artificial intelligence and other technologies.

 

\chapter{Introduction}

\begin{itemize}
    \item  Internet Digital Economy
   \paragraph{}  The level of digitalization and intellectualization of information benefiting the people has
been continuously improved, and the network poverty alleviation has achieved remarkable results
[2]- Government services such as health code and communication big data travel card have helped
epidemic prevention and control, and made great contributions to China’s epidemic prevention. In
the future, digital technology and Internet of things technology will continue to serve the fight
against the epidemic and economic recovery. In the process of anti-epidemic, information
technology and Internet of things technology play an important role in epidemic surveillance and
analysis, virus source query, community management application, virus detection and diagnosis,
vaccine production, supply and marketing and its deployment.
\item  Importance Of Digital Economy
\par Through the impact of the new coronavirus epidemic, more and more people realize the
importance of digital information and how to use digital information. The function of the digital
economy and technology to serve the social infrastructure is increasingly obvious. We can predict
that the digital economy and technology will bring greater changes to society in the future. People
can understand the way and technical indicators of digitization through technical means. At
present, artificial intelligence technology, big data, Internet of things and other new technologies
and new applications are gradually rising, triggering a new scientific and technological change.
Internet and Internet of things and other technologies have stronger power and wider space.
Digital economy industry and Internet of things technology perform well in the aspects of supply
and demand docking, resource allocation, industrial strategy upgrading, etc., which will be more
obvious in the future.In particular, driven by new theories and technologies such as mobile
Internet, big data, supercomputing, sensor networks and brain science, artificial intelligence will
present new features such as deep learning, cross-border integration, human computer
cooperation, open group intelligence and independent control.
\end{itemize}
\pagenumbering{arabic}



\

\chapter{Development Of Intelligent Internet Of Things}
\section{Digital world Internet of things}
\paragraph{} Technology is a combination of the real world and the digital world. In the world of
the Internet of things, every object is embedded with a class of sensors, has its own unique
online authentication identity, and can be identified and contacted by the outside. Because
all kinds of goods have their own intelligent identity, they can be cheap for the enterprises
in need of information in the society, improve the utilization rate of goods regulation in the
whole society, reduce the waste of goods, and improve social efficiency.

\section{ Intelligent Internet}
\paragraph{}
Today’s intelligent Internet of things includes not only fitness wearable devices and
household appliances, but also intelligent Internet factories and smart cities. These
intelligent objects can query the corresponding information online . The development of
intelligent Internet of things is also inseparable from the upgrading and improvement of
Internet software and hardware technology and the reduction of cost, as well as the
innovation and development of big data technology and cloud computing technology.
According to experts’ prediction, Internet of things equipment will continue to grow with
the progress of technology, at the same time, the output value of products and services
related to Internet of things technology will reach a new high.
\section{ Network technology}
\paragraph{}
Relying on the development of Internet of things technology, some famous network
technology companies are spending a lot of money to build Internet of things facilities and
equipment, and set up corresponding business departments. Some famous consumer goods
and manufacturers are also taking advantage of the Internet of things to develop new
intelligent products day and night. In addition, Internet enterprises have received the favor
of many venture capital companies, and have extended an olive branch to internet
intelligent companies. With the continuous progress of information technology and
Internet of things technology, competent computer professionals are especially popular
with Internet of things companies. Visionary IT professionals will enjoy challenging and
lucrative job opportunities.

\chapter{CATEGORIES OF AI}
 \section{Conventional AI }
\paragraph{} Conventional AI mostly involves methods now classified as
machine learning, characterized by formalism and statistical analysis. This is also known as
symbolic AI, logical AI, neat AI and Good Old Fashioned Artificial Intelligence (GOFAI).
Methods include: Expert systems: apply reasoning capabilities to reach a conclusion. An expert
system can process large amounts of known information and provide conclusions based on them.
Case based reasoning. Bayesian networks. Behavior based AI: a modular method of building AI
systems by hand.
\section{Computational Intelligence }
\par Computational Intelligence involves iterative
development or learning (e.g. parameter tuning e.g. in connectionist systems). Learning is based
on empirical data and is associated with non-symbolic AI, scruffy AI and soft computing.
Methods include: Neural networks: systems with very strong pattern recognition capabilities.
Fuzzy systems: techniques for reasoning under uncertainty, has been widely used in modern
industrial and consumer product control systems.
\newpage
\section{Evolutionary computation} 
\par Applies biologically
inspired concepts such as populations, mutation and survival of the fittest to generate 2
increasingly better solutions to the problem. These methods most notably divide into evolutionary
algorithms (e.g. genetic algorithms) and swarm intelligence (e.g. ant algorithms). its greatest
successes, albeit somewhat behind the scenes. Artificial intelligence is used for logistics, data
mining, medical diagnosis and many other areas throughout the technology industry
 
%\begin{table}[h]
%\caption{\bf Literature Review}
%\begin{tabular}{| l | c | c | l |}\\\hline
%Sr.No. & Authors & Title & Outcomes \\\hline
%1 & Sholk Gilda & Evaluating Machine Learning Algorithms for Fake News Detection & The system uses TF-IDF of bi-grams and PCFG for detection of fake news.\par It is found that TF-IDF of bi-grams fed into a Stochastic Gradient Descent model identifies non-credible sources with an accuracy of 77.2\%, with PCFGs having slight effects on recall. \\\hline
%\end{tabular}
%\end{table}
\newpage
\chapter{Typical problems to which AI methods are applied}
\paragraph{} Pattern recognition o Optical character recognition o Handwriting recognition o Speech
recognition o Face recognition Natural language processing, Translation and Chatter bots
Non-linear control and Robotics Computer vision, Virtual reality and Image processing. Game
theory and Strategic planning.
\section{Automation}
\paragraph{} Automation is the use of machines, control systems and information technologies to
optimize productivity in the production of goods and delivery of services. The correct incentive
for applying automation is to increase productivity, and/or quality beyond that possible with
current human labor levels so as to realize economies of scale, and/or realize predictable quality
levels. automation greatly decreases the need for human sensory and mental requirements while
increasing load capacity, speed, and repeatability.
\section{Cybernetics}
\paragraph{} Cybernetics in some ways is like the science of organization, with special emphasis
on the dynamic nature of the system being organized. The human brain is just such a complex
organization which qualifies for cybernetic study. It has all the characteristics of feedback, storage,
etc. and is also typical of many large businesses or Government departments. Cybernetics is that
of artificial intelligence, where the aim is to show how artificially manufactured systems can
demonstrate intelligent behavior.
\section{Hybrid intelligent system}
\paragraph{} Hybridization of different intelligent systems is an innovative
approach to construct computationally intelligent systems consisting of artificial neural network,
fuzzy inference systems, rough set, approximate reasoning and derivative free optimization
methods such as evolutionary computation, swarm intelligence, bacterial foraging and so on. The
integration of different learning and adaptation techniques, to overcome individual limitations and
achieve synergistic effects through hybridization or fusion of these techniques, has in recent years
contributed to an emergence of a large number of new superior classes of intelligence known as
Hybrid Intelligence.
\section{Intelligent agent}
\paragraph{} In artificial intelligence, an intelligent agent (IA) is an autonomous entity
which observes through sensors and acts upon an environment using actuators (i.e. it is an agent)
and directs its activity towards achieving goals.
\section{Intelligent control}
\paragraph{} Intelligent Control or self-organizing/learning control is a new emerging
discipline that is designed to deal with problems. Rather than being model based, it is experiential
based. Intelligent Control is the amalgam of the disciplines of Artificial Intelligence, Systems
Theory and Operations Research. It uses most recent experiences or evidence to improve its
performance through a variety of learning schemas, that for practical implementation must
demonstrate rapid learning convergence, be temporally stable, and be robust to parameter changes
and internal and external disturbances.
\section{Automated reasoning}
\paragraph{} The study of automated reasoning helps produce software that allows
computers to reason completely, or nearly completely, automatically. Although automated
reasoning is considered a subfield of artificial intelligence, it also has connections with theoretical
computer science, and even philosophy.
\section{Data mining}
\paragraph{} Data mining (the analysis step of the ”Knowledge Discovery in Databases”
process, or KDD), an interdisciplinary subfield of computer science, is the computational process
of discovering patterns in large data sets involving methods at the intersection of artificial
intelligence, machine learning, statistics, and database systems. The overall goal of the data
mining process is to extract information from a data set and transform it into an understandable
structure for further use.
\section{Behavior-based robotics}
\paragraph{} Behavior-based robotics is a branch of robotics that bridges artificial
intelligence (AI), engineering and cognitive science. Its dual goals are: To develop methods for
con- trolling artificial systems, ranging from physical robots to simulated ones and other
autonomous software agents To use robotics to model and understand biological sys- tems more
fully, typically, animals ranging from insects to humans. Cognitive robotics
\section{Developmental robotics}
\paragraph{} Developmental Robotics (DevRob), sometimes called epigenetic
robotics, is a methodology that uses metaphors from neural development and developmental
psychology to develop the mind for autonomous robots. The program that simulates the functions
of the genome to develop a robot’s mental capabilities is called a developmental program.

\newpage
\chapter{The combination of artificial intelligence and Internet of things}
\paragraph{}The combination of artificial intelligence technology and Internet of things equipment has brought
a new technological revolution. The two complement each other and give full play to their data
and technical advantages, which will bring a new experience to people’s life. If the Internet of
things wants to maintain sustainable growth, it must cooperate with artificial intelligence, because
the improvement of intelligent technology of the Internet of things is endless. The combination of
Internet of things and artificial intelligence generates intelligent Internet of things, which makes
the Internet of things have the ability of data exchange, and can use artificial intelligence
technology to transform data information into value information. The combination of artificial
intelligence and Internet of things will have an impact on the following aspects.
\section{Impact on business intelligence}
\paragraph{}Using artificial intelligence, we can analyze the big data in
the Internet of things and obtain effective information. The analysis of good data can prove the
business of the enterprise, understand where the enterprise is doing better and where it is not
doing well, and enable the enterprise to better find the path of improvement according to the data .
If enterprises can combine artificial intelligence and Internet of things, their efficiency and
decision-making power will be greatly improved, and their innovation and productivity will be
improved
\section{ Impact on industrial production}
\paragraph{} Through machine learning of artificial intelligence, we can
predict the possible problems of industrial Internet of things products, greatly increase the success
rate of new product development, and save production time and cost. In addition, the application
of artificial intelligence can improve and speed up the production process of subject.
\newpage
 \chapter{APPLICATIONS OF AI}
\paragraph{}   Artificial intelligence has been used in a wide range of fields including medical diagnosis, stock
trading, robot control, law, scientific discovery and toys.
\section{Hospitals and Medicine}
\paragraph{}
A medical clinic can use artificial intelligence systems to organize bed
schedules, make a staff rotation, and provide medical information. Artificial neural networks are used
as clinical decision support systems for medical diagnosis, such as in Concept Processing technology
in EMR software. Other tasks in medicine that can potentially be performed by artificial intelligence
include: Computeraided interpretation of medical images. Such systems help scan digital images, e.g.
from computed tomography, for typical appearances and to highlight conspicuous sections, such as
possible diseases. A typical application is the detection of a tumor. Heart sound analysis.
\section{Heavy industry}
\paragraph{} Robots have become common in many industries. They are often given jobs that
are considered dangerous to humans. Robots have proven effective in jobs that are very repetitive
which may lead to mistakes or accidents due to a lapse in concentration and other jobs which humans
may find degrading.
\section{Game Playing}
\paragraph{} This prospered greatly with the Digital Revolution, and helped introduce people,
especially children, to a life of dealing with various types of Artificial Intelligence. You can also buy
machines that can play master level chess for a few hundred dollars. There is some AI in them, but
they play well against people mainly through brute force computation–looking at hundreds of
thousands of positions. The internet is the best example where one can buy a machine and play
various games.
\newpage
\chapter{FUTURE SCOPE OF AI}
\paragraph{}
In the next 10 years technologies in narrow fields such as speech recognition will continue to improve
and will reach human levels. In 10 years AI will be able to communicate with humans in unstructured
English using text or voice, navigate (not perfectly) in an unprepared environment and will have some
rudimentary common sense (and domain-specific intelligence). We will recreate some parts of the
human (animal) brain in silicon. The feasibility of this is demonstrated by tentative hippocampus
experiments in rats. There are two major projects aiming for human brain simulation, CCortex and
IBM Blue Brain. There will be an increasing number of practical applications based on digitally
recreated aspects human intelligence, such as cognition, perception, rehearsal learning, or learning by
repetitive practice. The development of meaningful artificial intelligence will require that machines
acquire some variant of human consciousness.Without these uniquely human characteristics, truly
useful and powerful assistants will remain a goal to achieve. To be sure, advances in hardware,
storage, parallel processing architectures will enable ever greater leaps in functionality Systems that
are able to demonstrate conclusively that they possess self awareness, language skills, surface,
shallow and deep knowledge about the world around them and their role within it will be needed
going forward.
\chapter{CONCLUSION}

 \paragraph{} At present, we live in an environment surrounded by data, people have been very dependent on the
Internet and Internet smart devices. Every day, everyone will produce a lot of data, but these data have
not been well used by us. If Internet of things devices are not limited by networking devices, it is
difficult to estimate the amount of data they collect. It can be said that they can collect massive data.
The data they generate and use are very large. So much data provides a lot of opportunities and
convenience for people’s production, life and social activities [10]. Due to the limitation of conditions,
only a part of enterprises or people can make full use of the Internet of things to generate and store
data for production and life. Although a large amount of data is produced in production and life, it
cannot be analyzed and processed timely and effectively, and it does not make the data produce real
efficiency.

\newpage
\renewcommand\bibname{References}
\bibliographystyle{IEEEtran}
\begin{thebibliography}{1}

\addcontentsline{toc}{chapter}{References}
\bibitem[1]{b} Saito Kazumi,”Multiple topic detection by Parametric Mixture Models (PMM) “, Automatic web
page categorization for browsing. NTT Technical Review, 2005, pp.4-7.
\bibitem[2]{} Roy Deb.,” Grounded spoken language acquisition”, Experiments in word learning. IEEE
Transactions on Multimedia, 2003, pp.164-183.
\bibitem[3]{} Zhang R G, Hu X H, and Zong Y S,” Discretization of continuous attributes based on improved
discrete particle swarm optimization”, Computer Engineering and Applications,2017, pp.110-126.
\bibitem[4]{} Sang Y, Li K Q, and Yan D Q,” A data discretization algorithm based on improved chi-square
statistic’”, Journal of Dalian University of Technology,2012, pp.354-365.
\bibitem[5]{} Yu Zhanqiu,” Big data clustering analysis algorithm for internet of things based on K-means”,
International Journal of Distributed Systems and Technologies, 2019, pp.5-9. (in Chinese).
\bibitem[6]{} Zhou Fei-Yan, Jin Lin-Peng, Dong Jun,” Review of Convolutional Neural Network”, Chinese
Journal of Computers, 2017, pp.1019-1031. (in Chinese).
\bibitem[7]{} Rehman M H U, Chang V, Batool A,” Big data reduction framework for value creation in
sustainable enterprises”, International Journal of Information Management, 2016, pp.817-828.
\bibitem[8]{} Zang Maolin,” Human Resource Management in the Era of Big Data”, Journal of Human Resource
and Sustainability Studies, 2015, pp.25-30. (in Chinese).
\bibitem[9]{} Wu, L. Y., Chen, P. Y., and Chen, K. Y.,” Why does Loyaltycooperation Behavior Vary Over
Buyer-seller relationship. Journal of Business Research”,2015, pp.1002-1028.



\end{thebibliography}

\end{document}
